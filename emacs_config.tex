% Created 2018-11-25 Sun 22:41
\documentclass[11pt]{article}
\PassOptionsToPackage{hyphens}{url}
\usepackage[utf8]{inputenc}
\usepackage[T1]{fontenc}
\usepackage{fixltx2e}
\usepackage{graphicx}
\usepackage{longtable}
\usepackage{float}
\usepackage{wrapfig}
\usepackage{rotating}
\usepackage[normalem]{ulem}
\usepackage{amsmath}
\usepackage{textcomp}
\usepackage{marvosym}
\usepackage{wasysym}
\usepackage{amssymb}
\usepackage{hyperref}
\tolerance=1000
\usepackage{listings}
\usepackage[margin=1in]{geometry}
\usepackage[x11names]{xcolor}
\hypersetup{linktoc = all, colorlinks = true, urlcolor = DodgerBlue4, citecolor = PaleGreen1, linkcolor = black}
\author{David Pritchard}
\date{\today}
\title{Emacs Configuration}
\begin{document}

\maketitle
\tableofcontents

% background color for code environments
\definecolor{lightyellow}{RGB}{255,255,224}
\definecolor{lightbrown}{RGB}{249,234,197}

% create a listings environment for elisp
\lstset{%
  language=Lisp,
  backgroundcolor=\color{lightyellow},
  basicstyle=\fontsize{10}{11}\fontfamily{pcr}\selectfont,
  keywordstyle=\color{Firebrick3},
  stringstyle=\color{Green4},
  showstringspaces=false,
  commentstyle=\color{Purple3}
  % frame=lines
}


\section{Configuration}
\label{sec-1}

TODO

\subsection{Basic Configuration}
\label{sec-1-1}

\subsubsection{Set up the package management system}
\label{sec-1-1-1}

There are several Emacs package archives that can be used to download and update
packages from.  Emacs comes bundled with a package manager called \texttt{package.el},
which does the job of connecting to these archives and handling package
downloads, managing dependencies, and so on.  Most notably, it includes the
functions \texttt{package-list-packages} and \texttt{package-install}.  See
\url{http://wikemacs.org/wiki/Package.el} for more details regarding \texttt{package.el}, and
see \url{https://emacs.stackexchange.com/q/268/} for a description of the various
package archives.

The following code was taken from
\url{http://cachestocaches.com/2015/8/getting-started-use-package/}.  Some open
questions about this code remain:
\begin{itemize}
\item What does \texttt{package-initialize} do?  Autoload?
\item What does the line \texttt{(setq package-enable-at-startup nil)} do?
\item How does \texttt{require} know where to find \texttt{package}?
\item Why do we need the call to \texttt{eval-when-compile}?
\end{itemize}

\lstset{language=Lisp,label= ,caption= ,numbers=none}
\begin{lstlisting}
(require 'package)
(setq package-enable-at-startup nil)
(add-to-list 'package-archives '("melpa" . "http://melpa.org/packages/"))
(add-to-list 'package-archives '("marmalade" . "http://marmalade-repo.org/packages/"))
(add-to-list 'package-archives '("gnu" . "http://elpa.gnu.org/packages/"))
(package-initialize)

(unless (package-installed-p 'use-package)
  (package-refresh-contents)
  (package-install 'use-package))

(eval-when-compile
  (require 'use-package))
\end{lstlisting}




\subsection{Redefine customization storage approach}
\label{sec-1-2}

Eamcs by default appends configuration information to your startup file.  For
example, if you use the "Easy customization interface" (through \texttt{M-x customize}
or by some other means), then your customizations are made permanent across
Emacs instances by using this approach.  There are other ways that
customizations can show up as well: for example \texttt{package.el} uses it to store a
list of user-installed packages.

However, there is something disagreeable about mixing user-written code and
automatically generated code.  Furthermore, since the startup file is tangled
from this Org file, we have no choice but to separate the automatically
generated code to prevent it from being overwritten every time we tangle a new
file.  This is what is done in the following code block.

\lstset{language=Lisp,label= ,caption= ,numbers=none}
\begin{lstlisting}
;; direct Emacs to reflect future changes by updating the custom file
(setq custom-file "~/.emacs.d/custom.el")
;; load the existing file
(when (file-exists-p custom-file)
  (load custom-file))
\end{lstlisting}
% Emacs 25.3.2 (Org mode 8.2.10)
\end{document}